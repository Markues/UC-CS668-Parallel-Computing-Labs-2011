\documentclass[12pt]{article}
\newcommand{\MyFullName}{Jon Wedaman, Jeremy Lavergne, Alex Padgett, Jeff Farris}
\newcommand{\MyLastName}{Wedaman, Lavergne, Padgett, Farris}
\RequirePackage{macroNorm}
\title{ Parallel Computing \\ Lab 2: Sieve of Eratosthenes }
\author{\MyFullName}
\date{ Due Monday, April 25th }
\begin{document}
\maketitle
\thispagestyle{empty}
\begin{center}
  %Description
\end{center}
\setcounter{page}{0}
\newpage

\def\thesection{\Roman{section}.}
\hfill \\
\section{ Background }

\paragraph{}The Sieve of Eratosthenes is an algorithm used to find prime numbers between 1 and an arbitrary $n$. 
We can naïvely model this in parallel by assigning each work item an enumerated value ($1\ldots n$) and ``sieving'' each multiple of these values until all that remains are the primes. 
However, parallelizing the sieve in this fashion presents a distinct problem when the upper bound reaches large numbers such as the bound given in the assignment, $2^{30}$.
The graphics card simply cannot allocate an array of appropriate size to hold all integers up to this bound. 
Clearly we need to break down the problem into discrete chunks.

We approach this problem by partitioning the initial array into evenly sized chunks which can be digested by the GPU bounded by the number of primes found instead of $n$. 

\section{ Optimizations }
\begin{enumerate}[1.]
\item \textbf{Using a bit array instead of whole integer arrays}. For all practical purposes, we found that a \texttt{uint4} array with some bitmasking allows us to compress the size of our sieve.

\item \textbf{Selecting out the common primes}. A great deal of time is spent marking off one of the most common primes: $2$. By changing the indexing scheme of our bits, we can select only odd numbers and compress our space by half at almost no added time overhead.

\item \textbf{Building sieve block instances with the GPU}. When switching from one block of sieve to a new one, our sieve used the host to build new blocks and select out the existing primes. This seemed repetitive, and we built some extra kernel code to build new sieve blocks and filter out found primes using the GPU
\end{enumerate}

\section{ Timings }

\begin{center}
    \begin{tabular}{r | c | c | c | c |}
	\hline
   	Optimization & Test1 ($\mu s$) & Test2 ($\mu s$) & Test3 ($\mu s$) & TestAvg ($\mu s$) \\
   	\hline
   	0            & 537.473 & 545.274 & 532.194 & 538.314  \\
   	1            & 528.693 & 528.804 & 524.491 & 527.329  \\
   	2            & 421.529 & 431.790 & 472.849 & 442.056  \\
   	3            & 401.231 & 411.245 & 398.912 & 403.796  \\
    \end{tabular}
\end{center}

\newpage
\section{ Implementation }

Implemented in Python with OpenCL. Attached are two files: \texttt{main.py} and \texttt{part1.cl}. These respectively contain the python code that sets up the environment and the C code that runs on the device.
 
% \lstinputlisting[language=Python,basicstyle=\footnotesize\ttfamily]{../main.py}

\end{document}
